\documentclass[12pt,letterpaper]{article}
\begin{titlepage}
\title{A REPORT ON QR CODE RESULTS PROCESSING SYSTEM}

\author{JEMBA YUSUF\\	REG NO:14/U/6567/EVE\\	STD NO: 214017353}
\end{titlepage}


\begin{document}
\pagenumbering{gobble}
\maketitle{}
\newpage
\pagenumbering{arabic}
\section{INTRODUCTION}
  In many secondary schools today, knowing the students’ grades is one the key aspects everyone desires, so that they can determine the student’s performance. However, this is hindered by many students trying to manipulate the grades by forging reports and lack of knowledge by the administrators on how to prevent that from happening. This has led to the student’s decline in performance and many parents losing confidence in some schools.   
  \subsection{Background}
   
  \paragraph{}Grading and reporting are relatively recent phenomena in education. In fact, grading and reporting were virtually unknown in schools in the United States. Throughout much of the nineteenth century most schools grouped students of all ages and backgrounds together with one teacher in one-room schoolhouses, and few students went beyond elementary studies. The teacher reported students' learning progress orally to parents, usually during visits to students' homes.
  
\subsection{Problem Statement}
Many secondary schools today are facing the problem of report duplication and at the same time analyzing the students’ performance is hard on the side of the school administration. The QR Based result processing system will provide the QR generator that will print QR code on every report of the student and also provides decision analysis graphs for each subject to guide the administration.
\subsection{Objectives}
\subsubsection{Main Objectives}
To develop a desktop application (QR Based result processing system) that will prevent report duplication by printing QR codes on every report and provide decision analysis graphs to guide the administrators on every subjects.
\subsubsection{Specific Objectives}
	To study the current system with the aim of identifying its weakness and generate requirements of the new system.\\
	To design a system model for the QR Based result processing system.\\ 
	To implement the designed system and produce a prototype that can be used to test whether all the requirements have been met.\\
	To test and validate the QR Based result processing system basing on the user requirements.\\
\subsection{Scope}
The system covers only the O level section and the intended users are the teachers and the administrators. The system is restricted to running on only windows operating systems starting from windows 7 and above.
Given the fact that most school that use automated grading system are based in urban areas like Kampala, Jinja and others, we decided to take our case study to be Makerere College School.
The QR Based result processing system is limited to only windows desktop which will be printing the reports with the QR code.
\subsection{Significance}
•	The QR based resulting system will help the administration and the teachers to grade students results easily by just inputting in the marks once using the system.\\
•	 It will also help to print out reposts automatically with QR code to avoid duplication.\\
•	The system will also help the administration in analyzing the performance of the students using decision analysis.\\
•	 it will help the teachers in sorting out students automatically using the system which can also solve the problem of missing marks.\\
\subsection{ Methodology}

The source of the data will be from teachers and school administration who are doing the grading of students and analyzing the general performance in all subjects.
The flow charts and unified modeling language diagrams will be used to mode entities from the data collected and show the relationship between different subsystems in the system.\\
\subsection{References}
[1]” Merriam-Webster”,27 01 2017
[online].Avilable:https://www.merriam-webster.com\\/dictionary/secondary school






\end{document}